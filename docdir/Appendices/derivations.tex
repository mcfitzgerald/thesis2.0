\section{Derivations}

\subsection{One-Site Receptor}
\label{app:klotz1}

Start with the ligand binding reaction

\begin{equation}
\ce{R + L <=> RL}\
\end{equation}

and define the equilibrium association constant

\begin{equation}
K_1 = \frac{\text{[RL]}}{\text{[R][L]}} \quad \equiv \quad \text{[RL]} = K_1 \text{[R][L]}
\end{equation}

define the fractional saturation \(\bar{\nu}\) 

\begin{equation}
	\bar{\nu} = \frac{\text{[bound ligand]}}{\text{[total receptor]}} = \frac{\text{[RL]}}{\text{[R]+[RL]}}
\end{equation}

use A.1.2 to substitute [RL] and factor out [R] to get

\begin{equation}
	\bar{\nu} = \frac{K_1 \text{[L]}}{1+K_1 \text{[L]}}
\end{equation}

\subsection{Multi-Site Receptors}

\subsubsection*{Two-Site}

For a two-site receptor, extend the one-site binding reaction with an additional ligand binding step

\begin{equation}
\ce{R + L + L <=> RL + L <=> RLL}\
\end{equation}

and define the second step equilibrium association constant

\begin{equation}
K_1 = \frac{\text{[RL]}}{\text{[R][L]}} \quad \text{and} \quad K_2 = \frac{\text{[RLL]}}{\text{[RL][L]}}
\end{equation}

add the new species to the fractional saturation definition including a factor of 2 (1 mole of RLL has 2 equivalents of L)

\begin{equation}
	\bar{\nu} = \frac{\text{[RL]} + 2 \text{[RLL]}}{\text{[R]} + \text{[RL]} + \text{[RLL]}}
\end{equation}

substitute [RL] and [RLL] using A.2.2, factor out [R], and normalize with a factor of \(\frac{1}{n}\), where \(n = \) the number of binding sites, in this case: 2 (this is required so that \(\bar{\nu}\) takes on values between 0 and 1)

\begin{equation}
	\bar{\nu} = \left(\frac{1}{2}\right) \frac{K_1 \text{[L]} + 2 K_1 K_2 \text{[L]}^2}{1 + K_1 \text{[L]} + K_1 K_2 \text{[L]}^2}
\end{equation}

\subsubsection*{\emph{n}-Sites}

The above approach can be used to derive the binding equation for \(n\) binding sites

\begin{equation}
	\bar{\nu} = \left(\frac{1}{n}\right) \frac{K_1 \text{[L]} + 2 K_1 K_2 \text{[L]}^2 + \cdots + n K_1 K_2 \cdots K_n \text{[L]}^n}{1 + K_1 \text{[L]} + K_1 K_2 \text{[L]}^2 + \cdots + K_1 K_2 \cdots K_n \text{[L]}^n}
\end{equation}

\subsection{Criteria for Cooperativity}

\subsubsection*{Cooperativity in two-site sytems}

Consider a system in which there are two one-site receptors in equal proportion

\begin{equation}
	2  \bar{\nu} = \frac{K_a \text{[L]}}{1+K_a \text{[L]}} + \frac{K_b \text{[L]}}{1+K_b \text{[L]}}
\end{equation}

combining their separate fractional saturation equations by cross-multiplying yields

\begin{equation}
	2  \bar{\nu} = \frac{(K_a + K_b)\text{[L]} + 2 K_a K_b \text{[L]}^2}{1 + (K_a + K_b)\text{[L]} + K_a K_b \text{[L]}^2}
\end{equation}

substituting this with the assignments \((K_a + K_b) = K_1\) and \(K_a K_b = K_1 K_2\) returns the binding equation for a receptor with two sites (A.2.4)

If the two receptors have equivalent receptor sites, i.e., \(K_a = K_b = k\), then \(K_1 = 2k\)




\clearpage
