\documentclass{article}
\usepackage{mathtools}
\usepackage{parskip}
\usepackage{graphicx}
\usepackage[flushmargin]{footmisc}
%\usepackage{fontspec}
\usepackage{microtype}
\numberwithin{equation}{section}
\frenchspacing
%\setmainfont{Hoefler Text}
\begin{document}
\renewcommand{\thefootnote}{\fnsymbol{footnote}}

\section{Introduction}

\textsc{Author's Note}: The following section contains what I hope is the essential background to get an unfamiliar reader up to speed and provide context. The many nuances will be developed in detail further in the text. Several treatises are available on the topic of ligand-binding and cooperativity, and the reader is directed to some for more in-depth background [klotz, wyman gil].

\subsection{Ligand Binding and Cooperativity}

The complex biochemical processes that give rise to life depend on the recognition of one molecule by another: a hormone binding to its receptor, an enzyme binding its substrate, a transcription factor binding \textsc{dna}, etc. A receptor recognizes its ligand, conventionally the smaller molecule, through non-covalent interactions comprising electrostatic forces, van der Waals forces, and the hydrophobic effect. The degree of recognition or \emph{affinity} of the receptor for its ligand is determined by the strength and number of non-covalent interactions between them, and those in turn are determined by the structure and chemical composition of the molecules, e.g., the type and spatial configuration of amino acids in a receptor protein. Analyses of ligand-receptor systems focus on affinity as the system's characteristic property and quantify it with the equilibrium constant of the binding reaction.


Differences in the magnitude of affinity are why a given receptor binds one ligand versus another. For example, an estrogen receptor will bind the hormone estrogen and subsequently effect a cellular change, but that same receptor will be unaffected by a another hormone such as insulin. This is because the structure and composition of the binding site on the receptor are complementary to those of the ligand and enable the formation of sufficient non-covalent interactions. Estrogen and insulin differ significantly in structure and composition (one is a steroid and the other a peptide) hence insulin cannot interact favorably with the binding site on the estrogen receptor. The converse is also true that a ligand could be contrived to have a greater affinity if it can participate in additional or stronger interactions.\footnote{This is the \emph{de facto} strategy used in drug discovery and development, particularly of those that function as competitive inhibitors.} This logic can be applied to receptors as well, where structural and compositional differences of binding sites account for differences in affinity. These implications become important when recalling that a protein has conformational flexibility and can take on structural changes.

Consider a receptor with two binding sites for the same ligand. If the sites are structurally and compositionally the same, then we would expect them to have the same affinity. But even if they are compositionally the same, if the structure changes, so will the nature of the interactions with ligand and therefore the affinity. When a ligand binds, it can induce a conformational change in a receptor, and when this results in a change in affinity at another compositionally identical site, it is called cooperatitivity. It works in both directions: affinity can increase (positive cooperativity) or decrease (negative cooperativity). This is often observed in multimeric receptors where there is an identical site on different subunits of the protein. The classic example of cooperativity is hemoglobin

For receptor proteins involved in cell signaling, modulation is an important function (capability?). Variations in affinity under different conditions 

This can be extended across multiple ligands or sites...

 












\end{document}